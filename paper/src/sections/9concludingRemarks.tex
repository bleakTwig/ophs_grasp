In this paper, the hotel selection variant of the orienteering problem is reviewed and two of the solutions found about the problem are closely examined, the Skewed Variable Neighbourhood Search method formulated in the original publication for the problem~\cite{divsalar2013}, and a Memetic Algorithm proposed by the same author a year later~\cite{divsalar2014}. Apart from this, the original orienteering problem is shortly explained, along with the team orienteering problem and the travelling salesman problem with hotel selections, due to the lack of publications regarding the OPHS by itself and how similar these three problems are to it. Algorithms for the OP, TOP and TSPHS are also reviewed, and a short description on how to adapt these algorithms for the OPHS is given as well.

An important factor about the OPHS is how the Total Number of Feasible Sequences of Hotels (TNFS) affects the execution time of any proposed algorithm to solve it. The maximal value of it for any instance is naturally equal to $H^{T-1}$, but the actual value can be much lower, and can be found using the initialization technique of the SVNS proposed by~\cite{divsalar2013}. Some graphs are given by~\cite{divsalar2014}, comparing the execution time and the average gap between the solutions found and the optimal solutions with the TNFS.

Another point worth noting is that, even if simple to develop, as of the current date nobody has published a simple multi-level greedy algorithm to find solutions for the problem. This algorithm would be beneficial to the research of the OPHS, since it would provide easy to compute solutions for any given instances as a benchmark to compare other algorithms too. Another point to consider is that no heuristic has challenged the problem that didn't use a multi-level approach, like the P-LS heuristic that managed to get astounding results for the TSPHS in incredibly low computation times due to how it didn't approach the problem in a multi-level fashion.

Many factors from other publications regarding the problem were taken into consideration while developing the algorithm for solving it. It is also worth noting that, while not very advanced, this is the first GRASP algorithm developed to tackle the OPHS, and with some more polishing it could potentially compete with the current best algorithms in the academy.

For future works concerning the problem, it is proposed to extend the OPHS with usual extensions of the OP, like time windows, route scores or various vehicles. Another simple extension would be to add a score to each hotel based on facilities, price or place. The development of a fast algorithm to calculate the TNFS for each instance would also be beneficial for the development of more research concerning the OPHS. Apart from this, an study on the problem to find its integer friendliness~\cite{revelle1993} would also be beneficial to improve the Mixed-Integer Linear Programming (MILP) algorithms developed to tackle it.

For future work concerning the algorithm, its improvement is the first thing that comes to mind. The tour Greedy Randomized Construction phase could be greatly improved by investing more computing time in pre-calculating how useful each trip could be before actually building them, along with the addition of a new Repair phase to work on fixing the currently scrapped solutions mentioned in Section 7. Apart from this, more movement will be added to the Local Search phase to have it explore more neighborhoods, and a path-relinking heuristic will be added to it to further improve the solutions using information found on solutions from previous iterations. Finding a non-multi-level approach to the problem is also considered for future work, considering the success of the P-LS heuristic mentioned before for the TSPHS.