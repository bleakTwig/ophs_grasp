The Orienteering Problem with Hotel Selection (OPHS) is a variant of the Orienteering Problem (OP) (Originally named Selective Travelling Salesman Problem)~\cite{tsiligirides1984}~\cite{laporte1990} introduced in 2013 by Divsalar, Vansteenwegen and Cattrysse~\cite{divsalar2013}. Like in the OP, its objective is to maximize the score obtained by visiting a given set of vertices, but it differentiates itself from the original by also considering hotels along the path where the traveler can rest.

As a generalization of the OP, the OPHS is also NP-hard and is considered harder than the original because the start and end locations for each trip need to be optimized as well as the general tour. The selections of hotels in-between the tour is considered important for adding realism into the OP, since this is a strong factor when deciding a tour to be followed in the real world. As described by Vansteenwegen et al.~\cite{vansteenwegen2012}, the cost of an extra day's wage for a driver is much higher than the cost of driving a few extra kilometers, hence a tour with less trip should be preferred over a tour with a lower total travel time.

Another problem that is closely related to this one is the Traveling Salesperson Problem with Hotel Selection (TSPHS), formulated by P. Vansteenwegen, W. Souffriau and K. S\"orensen~\cite{vansteenwegen2012} but, since the TSPHS is based on the Travelling Salesman Problem (TSP) instead of the OP, there is no score associated to each customer because all customers must be visited, therefore the choice of the initial and last hotel is irrelevant.

A common example and practical application for the problem is that of the tourist who wants to select the best attractions while also choosing the hotels where he'll be staying during his visit. More practical applications include that of a submarine performing a surveillance mission, where various vertices are visited and resting points are required, or that of a traveller who wants to visit different landmarks but must stop for gas before his tank is empty.

In Section 2, the problem is more rigorously defined and some important factors for the development of an algorithm to find solutions are noted. Then, the mathematical formulation for the OP and the OPHS are given in Section 3, and in Section 4 the state of the art is explored and algorithms for some related problems are explained. The paper changes its focus in Section 5, where the representation used for developing an algorithm to solve the problem is given, and in Section 6 this algorithm is thoroughly described. In section 7, different experiments to tune the parameters and assert the robustness of the algorithm were ran, and in Section 8 its performance is compared with the other algorithms available. The paper is later concluded in Section 9, where tips for future work are given.