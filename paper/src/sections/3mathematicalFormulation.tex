First and foremost, each variable and constant to be be used is stated to avoid confusion:
\begin{itemize}
    \item $t\epsilon \{0,..,T-1\}$ denotes the trips.
    \item $h\epsilon \{0,..,H-1\}$ denotes the hotels. $h=0$ and $h=1$ are reserved as the first and last hotels of the tour respectively.
    \item $p\epsilon \{H,..,H+P-1\}$ denotes the POI.
    \item $S_i$ denotes the score associated with the POI $i$.
    \item $b_{i,j}$ denotes the time consumed when moving from node $i$ to node $j$, whereas $B_t$ denotes the time budget of a given trip $t$.
    \item $x_{i,j,t}$ is the variable used for formulating the OPHS as a mixed-integer linear problem. If, in trip $t$, a visit to vertex $i$ is followed by a visit to vertex $j$, $x_{i,j,t}=1$. It is $0$ otherwise.
    \item $u_i$ represents the position of vertex $i$ in the tour. This variable is only used to denote the sub-tour elimination constraint.
\end{itemize}

The objective function tries to maximize the score collected by visiting each POI. It is worth noting that if one wanted to assign a score to the hotels as well as the POI, the second and third sums would need to start at $i=0$ and $j=0$.
\begin{equation}
    \text{max} \sum_{t=0}^{T-1} \quad \sum_{i=H}^{H+P-1} \quad \sum_{j=H}^{H+P-1} S_i x_{i,j,t}
\end{equation}

s.t.
\begin{equation}
    \sum_{i=1}^{H+P-1} x_{0,i,0} = 1
\end{equation}
This constraint ensures that the first trip of the tour starts on the initial hotel, or $h=0$.

\begin{equation}
    \sum_{i=0}^{H+P-1} x_{i,1,T-1} = 1
\end{equation}
Similarly to the first one, this constraint ensures that the last trip of the tour ends on the final hotel, or $h=1$.

\begin{eqnarray}
    \sum_{h=0}^{H-1} \sum_{i=0}^{H+P-1} x_{h,i,t} = 1 \quad \forall t\epsilon \{0,..,T-1\}\\
    \sum_{h=0}^{H-1} \sum_{i=0}^{H+P-1} x_{i,h,t} = 1 \quad \forall t\epsilon \{0,..,T-1\}
\end{eqnarray}
The contraints in (4) and (5) ensure that each trip starts and ends in a hotel respectively.

\begin{eqnarray}
    \sum_{i=0}^{H+P-1} x_{i,h,t} = \sum_{j=0}^{H+P-1} x_{h,j,t+1} \quad &\forall t\epsilon \{0,..,T-2\}\nonumber\\
    &\forall h\epsilon \{0,..,H-1\}
\end{eqnarray}
This ensures that the end hotel of the trip $t$ matches the initial hotel of the trip $t+1$

\begin{eqnarray}
    \sum_{i=0}^{H+P-1} x_{i,k,t} = \sum_{j=0}^{H+P-1} x_{k,j,t} &\forall t\epsilon \{0,..,T-1\}\nonumber\\
    &\forall k\epsilon \{H,..,H+P-1\}
\end{eqnarray}
These constraints secure connectivity inside of each trip.

\begin{equation}
    \sum_{t=0}^{T-1} \sum_{j=0}^{H+P-1} x_{i,j,t} \leq 1 \quad \forall i\epsilon \{H,..,H+P-1\}
\end{equation}
This ensures that every POI is visited at most once.

\begin{equation}
    \sum_{i=0}^{H+P-1} \sum_{j=0}^{H+P-1} b_{i,j} x_{i,j,t} \leq B_t \quad \forall t\epsilon \{0,..,T-1\}
\end{equation}
These constraints ensure that the time budget for each trip is not exceeded.

\begin{eqnarray}
    u_i - u_j + 1 \leq (P-1)(1-\sum_{t=0}^{T-1} x_{i,j,t})\nonumber\\
    \forall i,j \epsilon \{H,..,H+P-1\}
\end{eqnarray}
Finally, these are the sub-tour elimination constraints, which are based on the Miller-Tucker-Zemlin (MTZ) formulation of the TSP~\cite{miller1960}.

Due to the formulation of the problem, the calculation for the Feasible Region is trivial. Considering that the only decision variable used is $x_{i,j,t}$ and that there is a total of $T$ trips and $H+P$ total nodes, the Search Space or Feasible Region is:

\begin{equation}
    FR = (H+P)^2\cdot T
\end{equation}

It is worth noting that this Solution Space considers many unfeasible solutions, and that with a finer analysis the number of decisions that actually need to be made can be much lower.

%First, to find a maximum cap for the Feasible Region, every possible movement for the traveller is evaluated, even if the tours found are unfeasible. First of all, one hotel from the total of $H$ must be chosen to act as the starting point for the tour or as $h=0$. Next, one other hotel from the remaining $H-1$ should be picked as the ending point for the tour, or as $h=1$. After these two decisions, the traveller can move to any given node, including the last hotel, which would mean picking one node out of the $H+P-2$ remaining one plus the last hotel, so a total of $H+P-1$ decisions are possible. After each iteration, one less node remains in the available nodes to visit next until only the ending point of the tour is available. Considering all of these decisions, the cap for the feasible region, which considers many unfeasible solutions, would be:

%\begin{eqnarray}
%    FR &= (H\cdot (H-1))\cdot (H+P-1)!\nonumber\\
%    FR &= (H^2-H)\cdot (H+P-1)!
%\end{eqnarray}

Only one improvement was made from the original mathematical model proposed by Divsalar et al.~\cite{divsalar2013}, though it is minor. In the objective function, the hotels were removed from the second and third sum, since no score is collected on them, and thus they can be removed without issues. Apart from that, the notation and the domains of some of the variables were changed, but this was done to improve the general legibility of the model and does not provide any real improvement to the model itself.